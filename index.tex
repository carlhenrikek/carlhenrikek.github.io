% Created 2016-12-21 Wed 10:37
% Intended LaTeX compiler: pdflatex
\documentclass[presentation]{beamer}
\usepackage[utf8]{inputenc}
\usepackage[T1]{fontenc}
\usepackage{graphicx}
\usepackage{grffile}
\usepackage{longtable}
\usepackage{wrapfig}
\usepackage{rotating}
\usepackage[normalem]{ulem}
\usepackage{amsmath}
\usepackage{textcomp}
\usepackage{amssymb}
\usepackage{capt-of}
\usepackage{hyperref}
\usetheme{default}
\author{Carl Henrik Ek}
\date{\today}
\title{}
\hypersetup{
 pdfauthor={Carl Henrik Ek},
 pdftitle={},
 pdfkeywords={},
 pdfsubject={},
 pdfcreator={Emacs 25.1.1 (Org mode 9.0.1)}, 
 pdflang={English}}
\begin{document}

My name is \alert{Carl Henrik Ek} and I am a Lecturer in Computer Science at the University of Bristol, UK and a Docent in Machine Learning at the Royal Institute of Technology, Sweden.

My main research interest is in multi-view latent variable models. In specific I am interested in learning factorised models where the latent representation encodes some form of structure. I have been working on a large range of different applications, from computer vision and robotics to computational biology. Recently I have become interested in representations of human and animal behaviour, trying to create models which allows us to understand how our mental state effects our behaviour and trying to discover structures among individuals interacting. I try to focus on applications where data is scares such that understanding the data is necessary which means that we require “real” model of the data.

Short Bio
Before joining Bristol I was an Assistant Professor in Machine Learning at the Royal Institute of Technology (KTH) in Stockholm. I did my postdoctoral research at University of California at Berkeley as a member of Professor Trevor Darrell’s research group. My PhD is from Oxford Brookes University. I spent two years of my PhD at the University of Manchester where I was a research assistant in the Machine Learning and Optimisation group and a further six months at University of Sheffield as a visitor in the Machine Learning group. My supervisors were Professor Neil Lawrence and Professor Phil Torr. Prior to this I was at University of Bristol where I was working together with Dr. Neill Campbell on Computer Vision in specific related to natural image statistics. I hold a MEng degree in Vehicle Engineering from KTH in Stockholm.

\begin{frame}[label={sec:orgc6083b0}]{News}
\begin{description}
\item[{[2016-10]}] Awarded Teacher of the year in Computer Science at University of Bristol
\item[{[2016-06]}] Organising \href{http://iv2016.berkeleyvision.org/}{Workshop on Learning Representations at Intelligent Vehicles} in Gothenburg with Trevor Darrell and Erik Rodner. I am also a Associate Workshop Editor for the conference.
\item[{[2016-04]}] New paper titled Multi-view Learning as a Nonparametric Nonlinear Inter-Battery Factor Analysis available on \href{http://arxiv.org/abs/1604.04939}{arXiv}
\item[{[2016-02]}] Awarded the degree of Docent in Machine Learning at Royal Institute of Technology, KTH, Stockholm, Sweden
\item[{[2015-12]}] I gave a \href{https://www.ted.com/tedx/events/17751}{TEDx} talk titled “Why I do not fear Artificial Intelligence” you can find the Video of the talk \href{https://www.youtube.com/watch?v=g2FiOz9FlTc&list=PLsRNoUx8w3rNPudaJgiY_3OSFDVy79MU7}{here}.
\item[{[2015-12]}] Interview (in Swedish) for Campi magazine \href{https://campi.kth.se/nyheter/med-bergfast-tilltro-till-studenterna-1.613960}{URL}
\item[{[2015-11]}] Awarded Teacher of the year at Royal Institute of Technology, Sweden \href{file://bin/2015_TeacherOfTheYear.pdf}{motivation}
\item[{[2013-10]}] Interview (in Swedish) about teaching in national Swedish student magazine Shortcut \href{http://shortcut.se/artiklar/man-lar-sig-nar-man-har-roligt/}{URL}
\end{description}
\end{frame}
\end{document}
